\chapter{Problem analysis}
The chapter divides the analysis into four sections. 
The first section describes the scope of the improvement based on the mentioned championed issue recommended by \ac{LDT}. 
The second section mentions use cases that use the proposed improvement. 
The third section determines requirements based on the use cases, requirements given by proposing new language features, and Roslyn implementation internals. 
The last section describes the proposed language feature design, which is inspired by C\# language feature ideas mentioned in the previous chapter and validated by the requirements.

\section{Scope}

\section{Use cases}
\change{Describe how the feature is meant to be used}

\section{Requirements}
\change{Decribe requirements given by use cases}
\change{Describe requirements given by Roslyn}
\change{Describe requirements given by Future improvements}
\change{Describe requirements given by back compatibility}
\change{Describe requirements given by new language feature proposing}

\section{Language feature design}

\change{Discuss mentioned ideas}
\change{Choose suitable subset of them}
\change{Divide it as in the proposal}
\change{Explain why they sucsess the requirements}




