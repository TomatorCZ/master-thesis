\chapter{Evaluation}

The implementation and proposal is evaluated using the following metrics.

\section{Tests}

We run the same suite case before our changes and after our changes.
The results can be found in the \texttt{/attachment/TestResults} folder containg the \texttt{Before} folder and the \texttt{After} folder.
These folders contains results of the \texttt{-testCompilerOnly} suite case.
Except unimportant small changes in the compiler tests regarding adding new error messages caused by our examples, all tests which passsed before the improvement pass after the improvement,
This result ensures that the improvement doesn't break the compiler functionality in dramatic manner. 

\section{Demo examples}

The capibalities of the improvement is demonstrated in the following examples.
TODO: Describe
\begin{figure}[h]
\begin{lstlisting}[style=csharp, showstringspaces=false]
F4<_, _, string>(x => x + 1, y => y.ToString(), z => z.Length); 

static void F4<T1, T2, T3>(Func<T1, T2> p12, Func<T2, T3> p23, Func<T3, T1> p31) { }
\end{lstlisting}
\caption{Example 1}
\label{img73:example1}
\end{figure}
TODO: Describe
\begin{figure}[h]
\begin{lstlisting}[style=csharp, showstringspaces=false]
C2<int, B> temp3 = new C2<int, B>();

F6<I2<_, A>>(temp3);

class A {}
class B : A{}
interface I2<in T1, out T2> {}
class C2<T1, T2> : I2<T1, T2> {}
static void F6<T1>(T1 p1) {}
\end{lstlisting}
\caption{Example 2}
\label{img74:example2}
\end{figure}
TODO: Describe
\begin{figure}[h]
\begin{lstlisting}[style=csharp, showstringspaces=false]
C2<int, string> temp7 = new C2<int, string

F10<I2<_, _?>>(temp7);
\end{lstlisting}
\caption{Example 3}
\label{img75:example3}
\end{figure}
TODO: Describe
\begin{figure}[h]
\begin{lstlisting}[style=csharp, showstringspaces=false]
new C4(
   new C2<_>()  //C2<int>..ctor
);

class C1<T> {}
class C2<T> : C1<T> {}
class C4 
{
   public C4(C1<int> p1) {}
}
\end{lstlisting}
\caption{Example 4}
\label{img76:example4}
\end{figure}
TODO: Describe
\begin{figure}[h]
\begin{lstlisting}[style=csharp, showstringspaces=false]
new C5<_>(
    new C5<_>(1) //C5<int>..ctor(int)
);

class C5<T> : C1<T>
{
    public C5(T p1) {}
}
\end{lstlisting}
\caption{Example 5}
\label{img77:example5}
\end{figure}
TODO: Describe
\begin{figure}[h]
\begin{lstlisting}[style=csharp, showstringspaces=false]
new C3<_>( //C3<int>..ctor(C1<int>, int)
   new C2<_>(), //C2<int>..ctor()
1
);

class C3<T>
{
   public C3(C1<T> p1, T p2) {}
}
\end{lstlisting}
\caption{Example 6}
\label{img78:example6}
\end{figure}
TODO: Describe
\begin{figure}[h]
\begin{lstlisting}[style=csharp, showstringspaces=false]
F1(
new C9<_,_,int,_>(1) //C9<int, int, int, C1<int>>..ctor(int)
,
1
);

static void F1<T>(I2<T> p1, T p2) {}

class C9<T1, T2, T3, T4> : C1<T2> where T4 : C1<T3>
{
    public C9(T1 p1) {}
}
\end{lstlisting}
\caption{Example 7}
\label{img79:example7}
\end{figure}

\section{LDM meeting summary}