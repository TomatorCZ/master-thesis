\chapter*{Introduction}
\addcontentsline{toc}{chapter}{Introduction}

\info{Describe what is type inference.}
Statically typed languages have many advantages like revealing bugs in compilation time or performance.
To achieve these benefits, the languages demand type annotations from a programmer.
These type annotations define an actual type of variable during runtime protecting to make operations on incompatible data.
Because code usually contains a lot of variables whose type has to be known during compilation time, type inference was introduced to eliminate type annotations that can be deduced from a context.
Type inference tries to deduce a type of a variable using a context, where the variable is used.
That's used operations and interactions with other parts of the code.


\change{Describe type inference in C\#.}
C\# is a statically typed language whose type system ,besides common primitives and classes known from other languages, contains generics.
Generics are used for parametrizing types in order to create reusable code(e.g. containers).
C\# generics parametrizes types by other types.
Main feature of the C\# type inference is getting rid of type arguments in cases, where the arguments can deduces from a context.

\change{Compare it with type inference in Rust or Haskell as an example of Hindel-Millner type inference.}
\change{Describe Roslyn.}
\change{Mention CSharplang repo, community, and describe a process of accepting lang changes.}
\change{Give an overview of chapters.}