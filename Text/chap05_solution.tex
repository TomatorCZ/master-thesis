\chapter{Solution}

The solution consists of a proposal describing the language feature given in the previous chapter \ref{sect09:lang} and an implementation of the prototype in a separate Roslyn branch.

\section{Proposal}

The final version of the proposal can be found in the attachment as the\\ \texttt{Attachment/Proposal/partial-type-inference.md} file.

\subsection{Creation Process}

The proposal had three stages of development. 
The first version of the document was created in a personal repository \cite{online:personalRepo}, where it was reviewed by a member of Roslyn's development team. 
The review was in the form of a pull request \cite{online:personalPull} where the member suggested several changes on how to structure the proposal and how to refer to the original C\# standard documentation and pointed out possible improvements that would be beneficial to investigate.
\par
After the revisions were made, the member recommended to post it as a discussion \cite{online:discussion1}, which was the first time when a wider community could comment on the proposal. 
Besides several upvotes received from anonymous readers, another member of Roslyn's development team started to give his recommendations on how to adjust the document. 
The main change of the improvement was to erase most of the examples taken from the tests made together with the prototype and replace them with more references to the original C\# documentation.
\par
The third version of the improvement was published as the next discussion \cite{online:discussion2}, where it received even more emoticons as likes or hearts, which was a good sign of progress. 
At that time, the discussion contained just answers to the questions raised by the member of the Roslyn team, which clarified the intention of the improvement.
\par
After this step, the third stage was made by publishing the proposal as a pull request \cite{online:pull2}. 
This step was done after the recommendation from the team member. 
The pull request was continued by another round of clarifications, recommendations, and revisions from three members of Roslyn's team.
\par
The current stage of the proposal at the time of writing is that the pull
request is still open, waiting for the next requirements from the LDT.

\subsection{Content}

The whole document has two styles of describing the feature. 
The first style explains the intention of the improvement and necessary relations, which helps to understand it. 
The second style used in the detailed design section is rather a patch of C\# standard documentation, which enables improvement. 
So, the text doesn’t contain fluent sentences but fragments of the documentation that need to be changed.
\par
The proposal consists of five parts. 
The first part gives a quick overview of the proposed change, summarizing it in a few sentences.
\par
The second part describes the motivation why it should be done. The text
and the used examples are similar to those in section \ref{sect10:mot}.
\par
The third and largest part contains a detailed design of the improvement. 
There is a description of grammar change, where it explains a new underscore contextual keyword in the type argument list. 
It is followed by the change of binding method invocation and object creation expressions. 
This part describes the mentioned core design of the improvement with the changed method type inference algorithm. 
The design ends with extending compile-time checking dynamic member invocation, which is explained in section \ref{sect11:dynamic}.
\par
The fourth part comments on the reason for not doing other alternatives contained in the discussions. 
It also mentions two possible extensions of the improvement using the diamond operator \ref{sect13:ex2} and initializers \ref{sect12:ex1} in the type inference context.
\par
The last part suggests other potential improvements.

\section{Implementation}

The already mentioned proposal is tested by the implementation described in this section. 
The goal of the implementation is to observe the consequences of the proposed feature in a practical way, which can be tried by the C\# community. 
The goal will be achieved by contributing to the the \texttt{final/PartialTypeInference} branch \cite{online:roslynFork} of the forked Roslyn project on GitHub, which is public.
The copy of the branch containing the changes can be also found in the attachment in the \texttt{Attachment/Source/roslyn} folder. 
Since the proposal is in the state of probing the benefits, the implementation is rather a proof of concept than a ready-to-production code.

\subsection{Development Environment}

Since the compiler is a complicated program consisting of many parts, the creators provide a guide \cite{online:roslynGuide} describing a common workflow, including building the compiler, writing tests, and deployment.
\par
We started cloning the Roslyn repository \cite{online:roslynRepo} and opening it in an IDE.
Several IDEs can be used. We recommend Visual Studio 2022 \cite{online:vs}, which was used to implement the proposal. 
The IDE was chosen since the Roslyn folks have recommended it, and it has provided helpful static code analysis and a debugger, which is almost necessary when a programmer is unfamiliar with the code base. 
However, hardware requirements for opening the repository were quite high regarding memory consumption, as it required around 20 GB of RAM for a smooth experience with browsing and launching the code. 
So we recommend Visual Studio Code \cite{online:vsCode} in cases when the hardware resources are
limited as compensation for worse code analysis.
\par
However, the IDE is not required to launch or deploy the compiler.
The correct version of the .NET SDK specified in \texttt{global.json} placed in the project root is needed to build the project. 
We used SDK \texttt{8.0.1}, which can be downloaded from Microsoft's original websites \cite{online:sdk}. 
We also use the Windows operating system, although the deployment should also be possible on Linux or MacOS.
\par
Since building large programs like compilers can be complicated, the root folder provides three scripts that take care of it. 
The \texttt{Restore.cmd} script is run to download the required packages. 
The \texttt{Build.cmd} script is used to build the compiler, and the \texttt{Test.cmd} script is used to run tests of the compiler. 
The scripts have multiple options that modify the workflow. We will mention a few of them that were used during the implementation.

\subsubsection{Testing}

Launching the compiler tests is done by a sequence of Windows command console commands, as shown in Figure \ref{img65:runTests}. 
Since the repository also contains an implementation of a C\# extension to Visual Studio, tests are divided into several groups that test the independent parts of the compiler. 
In our case, we used the \texttt{-testCompilerOnly} flag to run basic compiler tests. 
The tests will be a baseline, which has to be passed by the implementation of the proposal. 
The result of \texttt{Build.cmd} command is a built compiler placed in the \texttt{roslyn/artifacts} folder. 
Test results is stored in the \texttt{roslyn/artifacts/TestResults} folder as \texttt{.xml} files and \texttt{.html} files for displaying in the browser.
\begin{figure}[h]
\begin{lstlisting}
Restore.cmd
Build.cmd
Test.cmd -testCompilerOnly
\end{lstlisting}
\caption{Running C\# compiler tests on Windows.}
\label{img65:runTests}
\end{figure}
\par
The results of the tests, when we started the implementation, can be seen in the attachment in the \texttt{Attachment/TestResults/Tests-Master} folder, where two tests regarding Visual Basic were not passed. 
Since our contribution is the C\# compiler, we ignore it.

\subsubsection{Deployment}

The standard deployment of a new compiler version is bundling it with a new version of the SDK, which would require compiling the SDK from sources. 
However, there is a possibility of injecting a custom version of the compiler into an already installed SDK. 
This option was created to temporarily hot-fix compiler problems delivered with an SDK. 
Since the implementation is a proof of concept, this option will be sufficient to check the main functionality of the proposal directly in the user’s code.
\par
The injection is done by referencing a special package generated by the\\ \texttt{Build.cmd} script in a project file of a target C\# project using partial type inference. The Nuget cache containing .NET packages needs to be modified before the package generation. 
The Nuget cache is usually placed in the\\ \texttt{C://Users/\%user\%/.nuget} folder on Windows, where it is required to delete the version of the \texttt{Microsoft.CSharp.Net.Compiler.ToolSet} package, which will be generated by the \texttt{Build.cmd -publish} command. 
The command generates the \texttt{roslyn/artifacts/publish/Shipping} folder containing the\\ \texttt{Microsoft.CSharp.Net.Compiler.ToolSet} package, which needs to be referenced. 
Additionally, the project file has to define a folder containing the package that will be used to restore the packages in the Nuget cache. 
The last required action is specifying a language version, which has to be set to \texttt{preview}, enabling the partial method type inference and the partial constructor type inference. 
Figure \ref{img68:csproj} shows a modified project file of the demo application, which can be found in the \texttt{Attachment/Demo} folder. 
The \texttt{PropertyGroup} element contains a definition of the language version and a folder for package restoration. 
The \texttt{ItemGroup} element contains a definition of the package reference injecting the locally built compiler.
\begin{figure}[h]
\begin{lstlisting}
<Project Sdk="Microsoft.NET.Sdk">
  <PropertyGroup>
    <OutputType>Exe</OutputType>
    <TargetFramework>net8.0</TargetFramework>
    <LangVersion>preview</LangVersion>
        <RestoreSources>
      E:\roslyn\artifacts\packages\Debug\Shipping
    </RestoreSources>
  </PropertyGroup>

  <ItemGroup>
    <PackageReference 
      Include="Microsoft.Net.Compilers.Toolset" 
      Version="4.9.0-dev"
    />
  </ItemGroup>
</Project>
\end{lstlisting}
\caption{\texttt{Demo.csproj}}
\label{img68:csproj}
\end{figure}
\par
Then, a common \texttt{dotnet build} command can be used to compile the project using a custom compiler version instead of the bundled one. 
An easy way to check the currently used compiler is by putting the \texttt{\#error version} pragma anywhere in the compiled code. 
Figure \ref{img66:pragma} shows the usage of the pragma in the demo application.
\begin{figure}[h]
\begin{lstlisting}
using System;

#error version

...
\end{lstlisting}
\caption{Begining of \texttt{Program.cs}.}
\label{img66:pragma}
\end{figure}
\par
When the project is compiled again, it will raise a compilation error announcing the current package version, as shown in Figure \ref{img67:output}. 
We can see that the code was compiled by the \texttt{4.9.0-dev} compiler version, and the language version was set to \texttt{preview}.
\begin{figure}[h]
\begin{lstlisting}
E:\Demo>dotnet build
MSBuild version 17.8.3+195e7f5a3 for .NET
  Determining projects to restore...
  All projects are up-to-date for restore.
E:\Demo\Program.cs(4,8): error CS1029: 
  #error: 'version' [E:\Demo\Demo.csproj]
E:\Demo\Program.cs(4,8): error CS8304: 
  Compiler version: '4.9.0-dev (<developer build>)'. 
  Language version: preview [E:\Demo\Demo.csproj]

Build FAILED.

E:\Demo\Program.cs(4,8): error CS1029: 
  #error: 'version' [E:\Demo\Demo.csproj]
E:\Demo\Program.cs(4,8): error CS8304: 
  Compiler version: '4.9.0-dev (<developer build>)'. 
  Language version: preview. [E:\Demo\Demo.csproj]
    0 Warning(s)
    2 Error(s)

Time Elapsed 00:00:09.19
\end{lstlisting}
\caption{Output of building \texttt{Demo.csproj}}
\label{img67:output}
\end{figure}

\subsection{Repository Overview}

Roslyn’s repository contains several projects besides the C\# compiler, like the Visual Basic compiler or Visual Studio extension. 
Figure \ref{img68:roslynFolder} shows relevant files and folders of the \texttt{roslyn} directory, which have been adjusted or are used by the implementation. 
The root directory contains already mentioned helper scripts for building and deploying the compiler, the \texttt{src} folder containing the code, and the \texttt{artifacts} folder generated by the build system.
It also contains the \texttt{Roslyn.sln} solution, which can be opened in Visual Studio, enabling a better user experience during code browsing.
\par
The artifacts contain the \texttt{TestResults} folder, which contains the already-mentioned test reports, and the \texttt{bin} folder, which contains the compiled code and packages. 
The package required to inject the compiled compiler can be found in the nested \texttt{Shipping} folder.
\par
There are three important folders nested in the \texttt{src} folder. 
The \texttt{Test} folder contains compiler tests that can be run using the \texttt{testCompilerOnly} flag. 
The \texttt{Core} folder contains a common Visual Basic and C\# compiler code. 
It provides a public API used to compile and analyze code. 
The \texttt{CSharp/Portable} folder contains the C\# compiler sources, modified by the proposal’s implementation.
\par
Describing all parts of C\# compiler sources is out of the scope of this work since it is an extensive program, and most of the parts are not influenced by the implementation. 
So, the work mentions only a small part of the compiler, which is necessary to understand implementation internals. 
The compiler’s source code is divided into multiple folders where most of the changes are in the \texttt{Binder} folder. 
The folder contains logic for binding AST to \textit{bound tree} mentioned in the Roslyn section \ref{sect04:roslyn}. 
The implementation of the type inference algorithm can be found in the \texttt{Semantics/OverloadResolution} folder, together with the whole mechanism for overload resolution of methods and constructors. 
The \texttt{BoundTree} folder contains an XML description of automatically generated nodes representing \textit{bound tree}. 
The \texttt{Errors} folder contains definitions of all C\# errors and a list of features that can be disabled and which contain an item for enabling partial type inference. 
The \texttt{FlowAnalysis} contains several control flow analyzers, such as a nullability analyzer that checks the nullability state of variables. 
The \texttt{Generated} folder contains generated nodes of \textit{bound tree}, and the \texttt{Symbols} folder contains nodes of the AST tree.
\begin{figure}[h]
\begin{lstlisting}
+roslyn
  +src
    +Compilers
      +Core
      +CSharp
       +Portable
         +Binder
           +Semantics
             +Conversions
             +OverloadResolution
        +BoundTree
        +Errors
        +FlowAnalysis
        +Generated
        +Symbols
          +Source
      +Test
  +artifacts
    +bin
    +packages/Debug
      +Shipping
    +TestResults
  -global.json
  -Roslyn.sln
  -Build.cmd
  -Restore.cmd
  -Test.cmd
\end{lstlisting}
\caption{Simplified view of the \texttt{roslyn} directory content.}
\label{img68:roslynFolder}
\end{figure}

\subsection{Test Suite}

Roslyn tests use the xUnit framework \cite{online:xUnit}, providing wide API for testing application functionalities in various scenarios. 
We used this framework for our own tests in the \texttt{PartialTypeInferenceTests.cs} file. 
An additional helper method, \texttt{TestCallSites}, was made to save repeating code, which initializes the compilation, compiles the provided source code as a string, and verifies diagnostics representing compiler warnings and errors. 
Since we are interested in information regarding inferred type arguments, the source code is augmented by comments representing asserts which the augmented symbols have to hold.
\par
Figure \ref{img69:test} shows an example of a test case that tests a scenario where an inferred type argument syntax hides a typename having the \texttt{\_}  identifier. 
We can see that the \texttt{new F1<\_>(1)} expression is tested by the following comment describing the desired resolved call. 
The second argument of the \texttt{TestCallSites} defines which symbols we assert. 
In this case, all object creation expressions are asserted. The last argument describes expected diagnostics, which the compiler should announce. 
This example requires a warning regarding notifying a user about possible conflicting class names with the \texttt{\_} contextual keyword.
\begin{figure}[h]
\begin{lstlisting}[style=csharp, showstringspaces=false]
[Fact]
public void PartialConstructorTypeInference_UnderscoreClass()
{
  TestCallSites("""
class P
{
  static void M() 
  {
    new F1<_>(1); //-P.F1<int>..ctor(int)
  }

  class F1<T> { public F1(T p) {} }
}

class _ {}
""",
    Symbols.ObjectCreation,
    ImmutableArray.Create(
      Diagnostic(ErrorCode.WRN_UnderscoreNamedDisallowed, "_")
        .WithLocation(11, 7)
    )
  );
}
\end{lstlisting}
\caption{Example of a test.}
\label{img69:test}
\end{figure}

\subsection{Parsing Inferred Type Arguments}

We didn’t change the Roslyn parser. 
Instead of that, we created the\\ \texttt{SourceInferredTypeSymbol} symbol representing the inferred type argument and postponed the \texttt{\_} type argument recognition to the binding phase. 
In this phase, we changed the \texttt{BindIdentifier} method in \texttt{Binder\_Expressions.cs}, which looks up the type symbol represented by the provided identifier. 
The change is propagated into subsequent calls, which handle various cases of identifier binding, such as generic type names or fully qualified names. 
If the method is called during the binding type argument list of a method invocation or object creation expression, we change the mode, which enables the binding of the \texttt{\_} identifier as an inferred type argument symbol.

\subsection{TypeInferrer}

Replacing the \texttt{MethodTypeInference.cs} file with the \texttt{TypeInference.cs} was the biggest source code update in the code. 
Since the old algorithm for type inference was primarily focused on generic method type inference, we rewrote it in a more generic way using a concept of type variables and the relations between them and the type symbols. 
Additionally, we used the \texttt{\#region} syntax to divide the source code into multiple parts referencing corresponding parts of the already mentioned type inference algorithm. 
We also added a public API, which adapts the old method type inference API to the new one. 
This helped us to not change the other interacting parts of the codebase too much. 
There is the \texttt{InferMethod} static method, which has a signature similar to the previous one and is an entry point for the partial method type inference. 
We also added the \texttt{InferConstructor} static method as an entry point for the partial constructor type inference.

\subsection{Binding Partial-inferred Method}

The partial-inferred method binding is contained in the \texttt{Binder\_Invocation.cs} and the \texttt{OverloadResolution.cs} files. 
The changes in the binder were mostly about an error recovery and raising potential warnings. 
An example of that is a warning raised during dynamic method invocation binding where if the inferred type argument is used, we notify the user about a potential error in runtime since runtime doesn’t support the partial type inference. 
The change in the overload resolution was to analyze the method group with used inferred type arguments and to use the mentioned API of the type inferrer.

\subsection{Binding Partial-inferred Object Construction}

Binding partial-inferred object construction was the most tricky in the implementation since there was no preparation for the type inference as in the previous case. 
Moreover getting type information from the target is complicated since the binding order is reversed to the binding direction.
\par
The first step was to change the \texttt{OverloadResolution.cs} file to infer the type defining the constructor candidate when it contains inferred type arguments. 
Then, we had to substitute the constructor signature with inferred type arguments and choose the constructor that best fits.
\par
During the implementation, the following problem occurred. 
Figure \ref{img70:alpha} shows a definition of the \texttt{M} class containing two constructors. 
The second constructor’s body contains an expression creating an object of the same type.
The inferred type argument of this expression should be the \texttt{T} type parameter of the class. 
However, to be able to do that, alpha renaming has to be applied to the inferring type arguments since the type of the \texttt{p1} parameter is the \texttt{T} type symbol, and the type variable representing the inferring type argument is also the \texttt{T} type symbol.
Without this modification the type inference failed since the algorithm thought that it could not find the exact type symbol for that type argument. 
From the type inference perspective, these two symbols are different, so we had to distinguish these cases by replacing the type parameters represented by type variables with an alpha-renamed type symbol representing the same symbol. 
The alpha-renamed symbol is different from the one received from the argument list. 
This prevented the mentioned problem, and the expression was properly bound. 
Note that this problem also occurs in the binding of method invocation when there is a recursive generic method in which a recursion call is inferred and uses the same type of arguments.
\begin{figure}[h]
\begin{lstlisting}[style=csharp, showstringspaces=false]
class M<T> 
{
    M(T p1, int p2) {}
    
    M(T p1) 
    {
        new M<_>(p1, 1);
    }
}
\end{lstlisting}
\caption{Problem regarding alpha renaming.}
\label{img70:alpha}
\end{figure}
\par
The second step was to add additional type information received from the type parameter constraints of the containing type.
\par
The last step was to enable a target-typed inference. 
We used a similar way as in the \texttt{new()} operator. We created a new\\ \texttt{BoundUnconvertedInferredClassCreationExpression} representing an object creation expression containing the inferred type. 
This expression is bound lately when a target type is known in conversions defined in the \texttt{Conversions.cs} file. 
Whenever an expression is assigned, these conversions are used to check if it is possible to convert the expression to the target and generate appropriate conversion if necessary. 
In our case, the appropriate conversion is to invoke the binding of the object creation expression again with the already known target type, which is used in the type inference. 
Also, the expression doesn’t have to be assigned. 
For these situations, the \texttt{BindToNaturalType} method is called to invoke the conversions in these scenarios to make additional expression processing. 
In our case, it is binding the expression without the target type. 
The conversions are a bottleneck for this process, and every assignment or a statement uses it to invoke the additional necessary operation on expressions.
\par
The previous paragraph is not completely true in the binding order of the expression without a target. 
At the time of creating the unconverted inferred class creation expression, we bind the expression without the target and save the diagnostics obtained during that. 
This prevents us from binding the same expression without targeting multiple times, as we can see in Figure \ref{img71:binding}. 
There is the \texttt{F(new M<\_>(1))}, which calls a generic method with an inferred object creation expression. 
At the time of binding the object creation expression, we bind it without the target but still represent it as an unconverted inferred class creation expression. 
In the time of binding the \texttt{F} method, the overload resolution invokes the inference for each overload.
Since the inference is influenced by a type of argument, we have to use the type of that expression.
In this case, we use an already bound object creation expression without a target, which enables type inference to infer the type argument of the \texttt{F} function call. 
\begin{figure}[h]
\begin{lstlisting}[style=csharp, showstringspaces=false]
F(new M<_>(1))

static void F<T>(M<T> p1);
static void F<T>(M<T> p1, T p2);

class M<T> 
{
    M(T p1) {}
}
\end{lstlisting}
\caption{Preventing multiple binding.}
\label{img71:binding}
\end{figure}
Also, there is an opposite scenario. 
Figure \ref{img72:targetBinding} shows an example where the first argument needs the target type to infer the expression. 
The argument doesn’t influence the type inference algorithm since the corresponding parameter doesn’t contain the inferred type argument. 
In this case, after the overload of the \texttt{F} method is chosen, there is a phase of applying conversions from the arguments to the parameters invoked by the \texttt{CoerceArguments} method. 
This phase binds the first argument with the already known type of the parameter as a target type.
We prohibit this behavior when the object creation expression is used in the type inference by generating a different type of conversion for that. 
Also, in this phase, we add the previously saved diagnostics depending on whether we bound the expression without the target. 
See \texttt{ObjectCreationConvertionWithTarget} and \texttt{ObjectCreationConvertionWithoutTarget} conversions, which were created for this purpose.
\begin{figure}[h]
\begin{lstlisting}[style=csharp, showstringspaces=false]
F(new M<_>(), 1)

static void F<T>(M<int> p1);
static void F<T>(M<int> p1, T p2);

class M<T> 
{
    M() {}
}
\end{lstlisting}
\caption{Target-typed binding.}
\label{img72:targetBinding}
\end{figure}
\par
As we can see, binding an expression with the target type has complicated implementation in Roslyn.

\subsection{Nullable Walker}

The last part of the code changes is in the \texttt{NullableWalker.cs} file, which provides a nullability analysis pass. 
The basic principle of the analysis is traversing the control-flow graph and rebinding the nodes of \textit{bound tree}, if necessary, according to the nullability. 
The rebound nodes are rewritten at the end of the pass.
\par
An important change was to modify the analysis of the bound object creation expression symbol. 
Based on the information stored in the previous binding, we know if it was target-typed. 
If it was target-typed, we postpone the binding in the same way as the \texttt{new()} operator. 
The partial constructor type inference is added in the process of rebinding the constructor, where we use saved data from the previous binding to repeat it in the context of nullability. 
After the expression binding, we rewrite the containing type if it was changed during the type inference by storing the update to the end of the nullability analysis pass.