\chapter{Problem analysis}
\info{Describe the chapter}
The chapter divides the analysis into four sections. 
The first section describes the scope of the improvement based on the mentioned championed issue recommended by \ac{LDT}. 
The second section mentions use cases that use the proposed improvement. 
The third section determines requirements based on the use cases, requirements given by proposing new language features, and Roslyn implementation internals. 
The last section describes the proposed language feature design, which is inspired by C\# language feature ideas mentioned in the previous chapter and validated by the requirements.

\section{Scope}
\info{Describe why we choose only a small part of the C\# type inference}
The previous chapter indicates that type inference is a complicated process, where even the current C\# method type inference is difficult to understand. 
Hence, the thesis chooses a small part of C\# where it improves and introduces the type inference and would be possible to reason about and implement in the scope of this text. 
The second reason for choosing a minor change is that introducing a completely new type inference in C\# would rather have an experimental result, which would have a smaller chance of getting into production, which is different from the intention of this work. 
However, some more extensive changes in the type inference will also be mentioned to outline possible obstacles to introducing them in the C\#.
\par
\info{Specify the focus on partial type inference}
The thesis focuses on the already-mentioned \textit{partial type inference}, which was recommended by a member of \ac{LDT} and has a chance to be discussed in \ac{LDM} and potentially accepted. 
Analysis of this improvement contains a consideration of existing ideas, their consequences on C\#, and their difficulties in implementing them in Roslyn. 
Additionally, the work describes the relation to the Hindley-Millner formalization to express the strength of the type inference in a formalized way, which can be further used to compare it with other kinds of type inference in different programming languages and which decides the theoretical boundaries of the C\# type inference.

\section{Use cases}
\change{Describe how the feature is meant to be used}

\section{Requirements}
\change{Decribe requirements given by use cases}
\change{Describe requirements given by Roslyn}
\change{Describe requirements given by Future improvements}
\change{Describe requirements given by back compatibility}
\change{Describe requirements given by new language feature proposing}

\section{Language feature design}

\change{Discuss mentioned ideas}
\change{Choose suitable subset of them}
\change{Divide it as in the proposal}
\change{Explain why they sucsess the requirements}




